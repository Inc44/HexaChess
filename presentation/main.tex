\documentclass[aspectratio=169, 10pt]{beamer}
\usetheme{metropolis}
\usepackage[T1]{fontenc}
\usepackage[french]{babel}
\usepackage[most]{tcolorbox}
\usepackage{fontawesome5}
\usepackage{lmodern}
\usepackage{tikz}
\usetikzlibrary{shapes.geometric, arrows.meta, positioning, shadows, calc, fit, backgrounds}
\definecolor{DarkGray-10}{HTML}{14191E}
\definecolor{SandyBrown}{HTML}{E8AB6F}
\definecolor{Peru}{HTML}{D18B47}
\definecolor{Red}{HTML}{ED3524}
\definecolor{Green}{HTML}{2EDA77}
\definecolor{Blue}{HTML}{016FFF}
\setbeamercolor{block title}{bg=Peru,fg=white}
\setbeamercolor{frametitle}{bg=DarkGray-10,fg=SandyBrown}
\tikzset{
	hex/.style={regular polygon, regular polygon sides=6, draw, thick, minimum size=1.2cm, inner sep=0pt},
	archnode/.style={
		rectangle, rounded corners, draw=DarkGray-10, fill=white, very thick,
		minimum width=3cm, minimum height=1.6cm, align=center, drop shadow={opacity=0.15}
	},
	dbnode/.style={
		cylinder, shape border rotate=90, aspect=0.25, draw=DarkGray-10, fill=white, very thick,
		minimum width=2cm, minimum height=1.8cm, align=center, drop shadow={opacity=0.15}
	},
	flow/.style={->, >=Latex, thick, draw=DarkGray-10}
}
\title{HexaChess}
\author{Inc44}
\begin{document}
{
\usebackgroundtemplate{
\begin{tikzpicture}[remember picture,overlay]
	\fill[DarkGray-10] (current page.south west) rectangle (current page.north east);
	\foreach \i in {0,...,12}
		\foreach \j in {0,...,8} {
			\pgfmathsetmacro{\xoff}{\i * 1.4 + mod(\j,2) * 0.7}
			\pgfmathsetmacro{\yoff}{\j * 1.2}
			\node[hex, draw=SandyBrown, opacity=0.08, rotate=30] at ($(current page.south west) + (\xoff cm, \yoff cm)$) {};
		}
	\node[opacity=0.1, text=white] at (current page.center) {\fontsize{100}{100}\selectfont \faChess};
\end{tikzpicture}
}
\begin{frame}[plain]
	\centering
	\vspace{1cm}
	\textcolor{SandyBrown}{\Huge \textbf{HexaChess}}

	\vspace{0.5cm}
	\textcolor{white}{\Large \textbf{Échecs Hexagonaux}}

	\vspace{0.5cm}
	\textcolor{white!80}{\large Architecture, Organisation \& Rétrospective}

	\vspace{1.5cm}
	\textcolor{white!60}{\small \today}
\end{frame}
}
\begin{frame}{Vue d'ensemble du projet}
	\begin{tcolorbox}[colback=SandyBrown!10, colframe=Peru, title=\textbf{\faBullseye\ Objectif}]
		Développer une version numérisée du jeu d'échecs de Gliński (plateau hexagonal de 91 cases) avec le mode multijoueur.
	\end{tcolorbox}

	\vspace{0.5cm}
	\begin{columns}[T]
		\column{0.48\textwidth}
			\textbf{\faCogs\ Fonctionnalités clés}
			\begin{itemize}
				\item[\textcolor{Green}{\faCheck}] Plateau hexagonal (\texttt{AxialCoordinate})
				\item[\textcolor{Green}{\faCheck}] Architecture client-serveur
				\item[\textcolor{Green}{\faCheck}] IA Minimax (Alpha-Beta)
				\item[\textcolor{Green}{\faCheck}] Succès, système Elo \& tournois
			\end{itemize}
		\column{0.48\textwidth}
			\textbf{\faCode\ Stack technique}
			\begin{itemize}
				\item[\faJava] \textbf{Core :} Java 21, Maven
				\item[\faDesktop] \textbf{UI :} JavaFX, GluonFX (Mobile)
				\item[\faDatabase] \textbf{Data :} MariaDB, JDBC
				\item[\faServer] \textbf{API :} HTTP Server natif
			\end{itemize}
	\end{columns}
\end{frame}
\begin{frame}{Architecture technique}
	\centering
	\resizebox{0.95\textwidth}{!}{
	\begin{tikzpicture}[node distance=1.5cm]
		\begin{scope}[on background layer]
			\fill[SandyBrown!10, rounded corners] (-2.2,-3.2) rectangle (3.2, 2.5);
			\node[anchor=north west, text=Peru, font=\bfseries] at (-2.2,2.5) {CLIENT (JavaFX)};
			\fill[Blue!10, rounded corners] (4.8,-3.2) rectangle (12.2, 2.5);
			\node[anchor=north west, text=Blue, font=\bfseries] at (4.8,2.5) {BACKEND (API \& DB)};
		\end{scope}
		\node[archnode] (ui) at (0.5, 0.5) {
			\textbf{\faLaptop\ UI Layer}\\
			\footnotesize \texttt{HexPanel}, \texttt{MainWindow}\\
			\textit{Rendu visuel}
		};
		\node[archnode] (logic) at (0.5, -2) {
			\textbf{\faBrain\ Game Logic}\\
			\footnotesize \texttt{State}, \texttt{Board}, \texttt{AI}\\
			\textit{Règles \& état}
		};
		\node[archnode] (server) at (6.5, 0) {
			\textbf{\faServer\ Server API}\\
			\footnotesize \texttt{Handlers}, \texttt{JWT}\\
			\textit{Validation \& auth}
		};
		\node[dbnode] (db) at (10.5, 0) {
			\textbf{\faDatabase}\\
			\textbf{MariaDB}\\
			\footnotesize \textit{Persistance}
		};
		\draw[flow, <->] (ui) -- (logic);
		\draw[flow, ->] (logic.east) -- node[above, font=\tiny, align=center] {\textbf{HTTP POST}\\JSON (Jackson)} (server.west);
		\draw[flow, <->, dashed] (logic.east) -- node[below, font=\tiny, align=center] {\textbf{Polling}\\(Sync Move)} (server.west);
		\draw[flow, <->] (server) -- node[above, font=\tiny] {JDBC / DAO} (db);
	\end{tikzpicture}
	}
\end{frame}
\begin{frame}{Répartition : logique \& données}
	\begin{columns}[T]
		\column{0.48\textwidth}
			\begin{tcolorbox}[colframe=Red, title=\textbf{\faUserAstronaut\ ... (Inc44) $\cdot$ Team Lead}]
				\footnotesize
				\textbf{Engine \& Server}
				\begin{itemize}
					\item Conception globale
					\item Moteur de rendu (\texttt{HexRenderer})
					\item Intelligence artificielle (\texttt{AI})
					\item Client-serveur (\texttt{API}, \texttt{Server})
				\end{itemize}
			\end{tcolorbox}
		\column{0.48\textwidth}
			\begin{tcolorbox}[colframe=Blue, title=\textbf{\faNetworkWired\ ... (Vincehhh) $\cdot$ Backend}]
				\footnotesize
				\textbf{Data \& Network Features}
				\begin{itemize}
					\item SQL (\texttt{Looping})
					\item DAO (\texttt{PlayerDAO}, \texttt{TournamentDAO})
					\item Entités (\texttt{Game}, \texttt{Player})
					\item Tournois, sons, édition de profil
				\end{itemize}
			\end{tcolorbox}
	\end{columns}
\end{frame}
\begin{frame}{Répartition : interface \& qualité}
	\begin{columns}[T]
		\column{0.48\textwidth}
			\begin{tcolorbox}[colframe=Peru, title=\textbf{\faMagic\ ... (gwenaelv) $\cdot$ Frontend}]
				\footnotesize
				\textbf{User Experience (UX)}
				\begin{itemize}
					\item Authentification (\texttt{Login}, \texttt{Register})
					\item Succès (\texttt{Achievement}, \texttt{Leaderboard})
					\item Mode puzzles
					\item Timer
				\end{itemize}
			\end{tcolorbox}
		\column{0.48\textwidth}
			\begin{tcolorbox}[colframe=Green, title=\textbf{\faBug\ ... (Lasaiden) $\cdot$ QA}]
				\footnotesize
				\textbf{Quality Assurance (QA)}
				\begin{itemize}
					\item Aide interactive (\texttt{Help})
					\item Tests unitaires (\texttt{Test})
					\item Validation
					\item Documentation d'aide
				\end{itemize}
			\end{tcolorbox}
	\end{columns}
\end{frame}
\begin{frame}{Conclusion}
	\centering
	\begin{columns}
		\column{0.42\textwidth}
			\begin{block}{\faCheckCircle\ État actuel}
				\small
				\begin{itemize}
					\item Moteur hexagonal stable.
					\item Multijoueur fonctionnel.
					\item Cross-platform (Desktop/Mobile).
					\item Persistance des données.
				\end{itemize}
			\end{block}
		\column{0.1\textwidth}
			\centering \textcolor{SandyBrown}{\Huge \faArrowRight}
		\column{0.42\textwidth}
			\begin{block}{\faRocket\ Perspectives}
				\small
				\begin{itemize}
					\item WebSockets (temps réel).
					\item Optimisation IA (transpositions).
					\item Internationalisation étendue.
					\item Correction de bugs
				\end{itemize}
			\end{block}
	\end{columns}

	\vspace{0.5cm}
	\textcolor{DarkGray-10}{\textbf{HexaChess}} est prêt pour la démonstration.

	\vspace{0.25cm}
	\footnotesize \faGithub\ \texttt{Inc44/HexaChess}
\end{frame}
\end{document}